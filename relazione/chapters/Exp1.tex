\chapter*{Esperimento 1}
\addcontentsline{toc}{chapter}{Esperimento 1}

\section*{Obiettivo}
Far lampeggiare un led ad una frequenza stabilita e accendere/spegnere un altro led in base alla pressione di un bottone.

\section*{Svolgimento}
Per far lampeggiare un led ad una frequenza stabilita utilizziamo un timer. 
Un timer è un componente hardware in grado di contare dei cicli di clock.

Il timer ha un counter che viene incrementato ad ogni ciclo del clock diviso per un prescaler; quando questo counter raggiungere il valore di auto-reload viene impostato un flag (UIF).

Sappiamo quindi che il flag UIF verrà settato ad una frequenza ricavabile dalla seguente formula:

\begin{displaymath}
 \text{Frequency} = \frac{CLK}{(PSC)*(ARR)}
\end{displaymath}

Sul reference manual troviamo che TIM6 è collegato al bus APB1. Andiamo a vedere qual è la frequenza del clock su questo bus e quindi calcoliamo PSC e ARR per avere una frequenza di 1 Hz.

\begin{displaymath}
 1 \si{Hz} = \frac{84\si{MHz}}{(20000)*(4200)}
\end{displaymath}

\begin{minted}
[
frame=lines,
framesep=2mm,
baselinestretch=1.2,
fontsize=\footnotesize,
]{C}
//set timer frequency
TIM6->PSC = 20000;
TIM6->ARR = 4200;
//enable counter
TIM6-> CR1 = TIM_CR1_CEN;
\end{minted}

Quando il registro CNT raggiungerà il valore di ARR verrà settato il bit UIF nel registro TIM\_SR.
Andiamo quindi nel loop infinito a controllare questo bit. Ogni volta che troveremo questo bit acceso andremo ad accendere o spegnere il led e resetteremo il flag.

Per controllare il led usiamo la periferica GPIO, utilizzandola come output. Useremo in particolare la periferica GPIOB. Nel registro ODR (output data register) troviamo 16 bit, corrispondenti allo stato (acceso/spento) di 16 pin.
Il nostro led è collegato al pin 0. Andiamo quindi a controllare se il led è acceso (ovvero se il bit corrispondente è a 1) e lo impostiamo a 0 o viceversa.

\begin{minted}
[
frame=lines,
framesep=2mm,
baselinestretch=1.2,
fontsize=\footnotesize,
]{C}
while(1){
//check timer flag, if timer got to arr
    if(TIM6->SR == TIM_SR_UIF){
	    //if led1 is on
		if(GPIOB->ODR & GPIO_ODR_OD0_Msk){
			//turn off led1
			GPIOB->ODR = GPIOB->ODR & ~GPIO_ODR_OD0_Msk;
		}
		else {
			//turn on led1
			GPIOB->ODR = GPIOB->ODR | GPIO_ODR_OD0_Msk;
		}
		//reset timer flag
		TIM6->SR = 0x0;
	}
}
\end{minted}

Per la gestione del bottone andiamo sempre ad utilizzare la periferica GPIO, questa volta come input.
In questo caso andremo a controllare il registro IDR (input data register) che contiene lo stato dei 16 pin connessi alla periferica GPIOC, in particolare andremo a guardare lo stato del pin 13.
Quando viene rilevata la pressione del bottone si cambia stato del secondo led.
È stato utilizzato un flag per capire se il bottone fosse tenuto premuto in quanto la velocità dell'esecuzione del loop infinito è maggiore del tempo che una persona ci mette a premere il bottone, e ciò risultava in un comportamento inaspettato del led.

\begin{minted}
[
frame=lines,
framesep=2mm,
baselinestretch=1.2,
fontsize=\footnotesize,
]{C}
while(1){
//if blue button in pressed
	if(GPIOC->IDR & GPIO_IDR_ID13){
		//check if button is not being continuosly pressed
		if (button_flag == 0){
			//if led2 is on
			if(GPIOB->ODR & GPIO_ODR_OD7_Msk){
				//turn off led2
				GPIOB->ODR = GPIOB->ODR & ~GPIO_ODR_OD7_Msk;
			}
			else {
				//turn on led2
				GPIOB->ODR = GPIOB->ODR | GPIO_ODR_OD7_Msk;
			}
		}		
		button_flag = 1;
	} else {
		button_flag = 0;
	}
}
\end{minted}